\documentclass[a4paper,12pt]{article}
\usepackage[utf8]{inputenc}
\usepackage{graphicx}


\begin{document}
	 \bibliographystyle{plain}
	\bibliography{bibliografia.bib}
\hspace{-0,75cm}
\textbf{Introdução}
\vspace{1,0cm}


Os sistemas offshore podem incluir dispositivos de superfície
posicionados acima da superfície da água, como em um navio ou plataforma, e
dispositivos submarinacilitar a perfuração ou a produção.

\vspace{1,0cm}
\hspace{-0,75cm}
\textbf{A importância da inspeção:}
\vspace{1,0cm}

Os dutos podem apresentar defeitos ao longo de suas vidas uteis devido a um ou mais
mecanismos de falha de material. Dentre esses pontos de falha podemos citar a corrosão, erosão e
abrasão que pode levar ao decaimento da espessura tanto da parede interna quanto da parede externa.
Se esse decaimento da espessura for expressivo, pode ocorrer vazamentos e em caso mais extremo a
ruptura por completo do duto

De acordo com Morris (1991), citado por Schäfer \& Pinto (2006), entre os anos de 1967 e 1990,
50\% dos problemas causados na malha de dutos dos Estados Unidos ocorreu pela corrosão, ocasionando
vazamentos.
Segundo a Petrobras o processo de inspeção é dividido em 2 zonas de acordo com o nível do mar, a parte emersa, e o trecho submerso.

A Região emersa, compreende a região acima do nível do mar, geralmente a inspeção é feita de forma visual e com auxilio de equipamentos, com o objetivo de identificar trincas, desgastes de espessura.O trecho submerso, comprende todas estruturas e equipamentos que estão abaixo do nivel do mar, ou na ZVM (Zona de variação da maré).Deve-se ter um programa de inspeção que contempla toda linha de duto de maneira visual, gráfica, ou quantitativa, referente ao desgates do mesmo.

\vspace{1,0cm}
\hspace{-0,75cm}
 \textbf{Patentes de inspeção}
 
 \vspace{1,0cm}
 \begin{center}
  \textbf{PI 0505622-5} 
 
 \textbf{Método de inspeção submarina de superfícies metálicas por ultra-som}
 \end{center}
 
 \textbf{Resumo:}
 
 Data da publicação:25/09/2007
 
 Empresa: Petrobras
 
 \vspace{1,0cm}
 
 A presente invenção se insere no campo das inspeções submarinas
de superfícies metálicas, com o uso de ultra-som para a medição da
espessura de placas, especialmente as que compõem os cascos das
unidades flutuantes.

O aumento da produção de petroleo é comum se ter uma maior utilização de FPSO (“\textit{Floating Production Storage and Offloading”}), todavia pelo constante uso dessas embarcações é comum ocorrerem problemas de corrosão na parte interna dos tanques de estocagem de óleo.

Essa corrosão é usualmente causada pelo contato das placas com o
ambiente corrosivo formado pelos microorganismos presentes na água de
injeção que vem misturada ao óleo cru, provenientes de reservatórios que
utilizam recuperação secundária.

Para evitar que corrosões por pit (do termo em inglês “pit
corrosion”) ocasionem o vazamento desses tanques para o mar, fato que
representa grande risco para o meio ambiente, entidades certificadoras
recomendam a inspeção dos cascos dessas embarcações a cada dois anos
e meio.

A medição da espessura de superfícies metálicas é comumente feita
pelo uso de equipamentos de ultra-som. Estes equipamentos contam com
dispositivos emissores de sinais acústicos que atravessam a superfície em
teste, e que ao serem refletidos na fronteira deste material, são captados
por sensores que registram o tempo de trânsito desses sinais. Com o tempo
de trânsito e a velocidade do sinal na superfície testada, o equipamento
calcula a espessura da placa inspecionada.

Existem outras patantes que utilizam a metodologia de inspeção por ultra-som, porém todas com limitações, seja a forma como é operada, por exemplo com o uso de mergulhadores, ou limitidas pelo procedimento a ser feito durante a inspeção, por exemplo, com a retirada da embarcação de armazenamento da água.


A presente invenção ensina um método para inspeção visual e ensaio não
destrutivo que utiliza um veículo de operação remota - ROV - equipado
com um aparelho de ultra-som para a medição da espessura de placas de
metal.
O veículo de operação remota utilizado pelo método como um veículo
de inspeção, faz uso de um ou mais sensores montados num suporte em
sua estrutura externa, que são responsáveis pela leitura dos sinais captados
na medição. O veículo de inspeção utilizado no método possui potência suficiente
para operar em alto mar, com precisão e segurança, não sendo necessário
que este seja dedicado somente à atividade de inspeção por ultra-som.

\vspace{1,0cm}
 \begin{center}
  \textbf{Pl 0603344-0} 
 
 \textbf{Equipamento para inspeção submarina de tubulações rígidas e método operacional.}
 \end{center}
 
 \textbf{Resumo:}
 
 Data da publicação:26/12/2007
 
 Empresa: Petrobras
 
 \vspace{1,0cm}

CAIO@ A presente invenção encontra seu campo de aplicação dentre os
equipamentos e métodos operacionais para inspeção submarina de
tubulações rígidas. Partícularmente, dentre os processos e equipamentos
para inspeção submarina de tubulações rígidas conectadas a uma
plataforma de produção de petróleo, conhecidas pelos especialistas como
“\textit{risers}”.

De uma forma geral, o petróleo obtido de reservatórios subterrâneos
é enviado a uma plataforma de produção por intermédio de uma tubulação
rígida, conhecida como "riser”. A tubulação rígida, além de imersa em um
meio altamente corrosivo, é submetida a uma movimentação constante
provocada pela movimentação da plataforma de produção, sujeita a
ondulações marinhas superficiais. Estas condições acabam por induzir
tensões e fadiga na tubulação, comprometendo a vida útil da mesma. Assim sendo, as soldas circunferenciais existentes entre os trechos
de tubulação apresentam maior susceptibilidade à ruptura por tensões e fadiga.

O equipamento para inspeção submarina de tubulações rígidas e
método operacional, objetos da presente invenção, contemplam uma
solução para a inspeção de tubulações rígidas interligando um poço a uma
plataforma de petróleo. Estas tubulações, conhecidas como "\textit{risers}”, estão
em constante movimento devido à ação de ondas e correntes marinhas.

Assim sendo, a presente invenção permite avaliar a integridade de
uma tubulação com as peculiaridades citadas acima por intermédio de um
equipamento mecânico de fixação, composto por dois braços mecânicos
que abraçam a tubulação a ser inspecionada. Um filme radiográfico é
instalado em um dos braços mecânicos enquanto que, em outro, é instalada
a sede de um disparador radioativo. Ambos os braços mecânicos são
conectados a um corpo central e a um mecanismo de fechamento, molas de
tração são responsáveis pela abertura e pelo fechamento do equipamento
mecânico de fixação.

O método de instalação do referido equipamento mecânico de fixação
compreende, dentre outras, as seguintes etapas:

- Descida de uma cesta contendo uma pluralidade de equipamentos
mecânicos de fixação, já com os filmes radiográficos devidamente
instalados, até o fundo do mar;

- Instalação de equipamentos mecânicos de fixação no trecho de
tubulação a ser inspecionado;

-Instalação de um disparador radioativo na sede própria localizada
em um dos braços mecânicos do equipamento mecânico de fixação;

- Acionamento do disparador radioativo e impressão no filme
radiográfico;

- Coleta dos equipamentos mecânicos de fixação e posterior análise
dos resultados.



\vspace{1,0cm}
 \begin{center}
  \textbf{US 6,904,818} 
 
 \textbf{Dispositivo de inspeção interna de \textit{riser}}
 \end{center}
 
  \textbf{Resumo:}
 
 Data da publicação:14/06/2005
 
 Empresa: Vetco Gray. inc
 
 \vspace{1,0cm}
 
 Nesta invenção as ferramentas utilizadas são aplicadasa analisando as estruturas por dentro da tubulação. Primeiramente um ultrasom é alocado numa unidade de inspeção, geralmente \textit{Crawler}, para determinar a espessura da parede, em seguida um segundo ultrasom é montado com intuito de analisar o volume das soldas da tubulação. Periodicamente o primeiro ultrasom emite um sinal acustico dentro do tubo, em diração ao eixo vertical, de forma que o mesmo identifica o sinal de retorno e determina a espessura da parede.
 

 Durante a inspeção da solda uma série de transducers são colocados para evitar "pontos cegos" na vistória. A unidade de inspeção rotaciona pelo tubo enquanto inspeciona as soldas, e ao mesmo tempo verifica a espessura.
 
 O dispositivo de inspeção utiliza, preferencialmente, cilindros pneumáticos para impulsionar os sapatos de aderência contra a parade interior do tubo.


Figura 2: É uma vista em corte transversal do dispositivo inspeção interna
localizado dentro de um tubo central de um riser, o dispositivo de inspeção
sendo montado de acordo com esta patente.

\vspace{1,0cm}
 \begin{center}
  \textbf{PI 0720394-2 A2} 
 
 \textbf{Robô articulado para inspeção ultrasônica a laser}
 \end{center}
 
  \textbf{Resumo:}
 
 Data da publicação:28/08/2008
 
 Empresa: Lockheed Martin Corporation
 
 \vspace{1,0cm}

A vigente invenção faz jus ao campo  de avaliações não destrutivas (NDE), e mais particulamente à performaces de NDE ultrasônica a laser de materiais a partir de uma unidade robótica.

As modalidades da presente invenção proporcionam um sistema de avaliação não destrutivo, ultrasônico operável para inspecionar um determinado material alvo. Esse sistema NDE inclui um robô articulado, uma cabeça de inspeção ultrasônica, modulo de processamento e um modulo de controle. A  cabeça de inspeção é operável para distribuir um feixe de laser de geração, um feixe de laser de detecção e uma luz modulada de coleta dispersa pelos materiais alvos.

O módulo de controle dirige o robô articulado para posicionar a cabeça de inspeção de ultrassom  de acordo com um plano de varredura determinado.

\begin{figure}[!h]
\centering
\includegraphics[width=6cm]{Figura_03.png}
\caption{Cabeça de inspeção aplicada a robô articulado.}
\label{fig:CL_logo}\end{figure}

Figura 3 Ilustra um robô articulado usado para posicionar uma cabeça de inspeção de ultrassom em relação ao material alvo.

\vspace{1,0cm}
 \begin{center}
  \textbf{BR 102019001625-6 A2} 
 
 \textbf{Dispositivo de inspeção semiautônomo de tubulações  de pequenos diametros}
 \end{center}
 
 \textbf{Resumo:}
 
 Data da Publicação: 04/08/2020
 
 Empresa: INSFOR LTDA
 
\vspace{1,0cm}

 O sistema é composto por um robô modular, que trabalha com câmera de alta definição na parte frontal, realizando a filmagem da tubulação, por meio do módulo de inspeção, que emprega cabeçotes convencionais para medir a espessura de paredes da tubulação, por exemplo, ultrassônicos (Sondas, duplo cristl, Phased array, ou TOFD. EMAT( Electromagnetic acoustic transducer), MFL (Magnetic Flux, Leakage) e ACFM (Alternating Current Field Measurement).
 
 A câmera frontal de alta definição é posicionada de forma a poder registrar toda a extensão da tubulação e capturar 360° da parte interna do tubo.
 
 O sistema do robô é composto por motores elétricos, controlados remotamente, com transmissão por conjunto de engrenagens, sendo o seu ajuste em relação ao diametro interno da tubulação feito por meio de molas auto ajustáveis, podendo trabalhar em tubulações entre 6,35cm e 12,7cm de diamatro.
 
 O sistema permite obter resultados em termpo real em longas distancias, sem necessidade de alargar a tubulação para realização das inspeções. Proporciona a inspeção visual em tempo real, permitindo  a integração de diversas técnicas para ensaios nao destrutidos nos equipamentos. As imagens são armazenadas em HD e podem ser transferidas para mídias móveis e discos virtuais.



\vspace{1,0cm}
 \begin{center}
  \textbf{BR 112019020686-8 A2} 
 
 \textbf{Disposição de sensor, veículo subaquático e método para detecção subaquatica de vazamentos em tubulações}
 \end{center}
 
 \textbf{Resumo:}
 
 Data da Publicação: 12/05/2020
 
 Empresa: Fugro Techonology
 
\vspace{1,0cm}

A presente patente se refere a uma disposição de sensores para detecção subaquática de um vazamento no corpo carregador de fluido, em que compreende uma estrutura de sustentação de sensor montável a um carregador para habilitar que a estrutura de sustentação de sensor possa ser movida e posicionada de modo adequado. 

Uma pluraridade de sensores térmicos é distribuida sobre a estrutura de sustentação de sensor que forma um arranjo de sensores. Os sensores são fornecidos pelas fibras ópticas conectáveis a um interrogador por meio de conectores de sinal. Cada fibra comprrende sensores de fibra optico intrísecos em respectivas localizações dentro da fibra, formando sensores térmicos. As fibras são dispostas sobre a estrutura de sustentação de modo que os sensores de fibra sejam distribuidos atráves da estrutura de sustentação.

\vspace{1,0cm}
\hspace{-0,75cm}
\textbf{A importância da limpeza}
\vspace{1,0cm}

A limpeza de tubulações e estruturas submersas é uma ação essencial para empresas do setor de óleo e gás, uma vez que esta é antecessora ao processo de inspeção. A limpeza constitui-se no ato de remover as incrustações marinhas que impredirão a visualização de eventuais defeitos durante a inspeção e que podem, eventualmente, alterar a carga sobre o qual o \textit{riser} está submetido.

Atualmente as operações de limpeza são realizadas por mergulhadores que raspam, com uma espatula, as incrustações marinhas, porém essas operações tem um custo muito elevado, além de por em risco a vida dos operadores. Com isso se faz necessario a busca e confecção de produtos que, de forma  autonôma e com utilização de escovas, espatulas, ou qualquer ferramenta de limpeza, possam realizar esta operação e torna-la mais seguras.

\vspace{1,0cm}
\hspace{-0,75cm}
 \textbf{Patentes de limpeza}
 
 \vspace{1,0cm}
 \begin{center}
  \textbf{BR 102014023816-6 A2} 
 
 \textbf{Aparelhagem de escovação para limpeza de \textit{riser}}
 \end{center}
 
 \textbf{Resumo:}
 
 Data da publicação:06/12/2016
 
 Empresa: Petrobras
 
 \vspace{1,0cm}
 
 A presente invenção refere-se a uma aparelhagem que atua na
limpeza da superfície externa do trecho vertical de uma linha de coleta de
produção, \textit{riser}. O mecanismo é remotamente controlado, e depois de acoplado ao riser realiza ciclos de limpeza por escovação, movendo-se preferencialmente ao longo do trecho em catenária livre até aproximadamente 60 metros de profundidade

O mecanismo é capaz de alterar o posicionamento das escovas
sobre a superfície externa do \textit{riser}, garantindo a atuação das mesmas em
torno de todo o perímetro do \textit{riser}. Além da limpeza, opcionalmente o dispositivo pode ser utilizado
para realizar a inspeção simultânea externa dos \textit{risers}.

 A presente invenção se refere a uma aparelhagem composta
basicamente por um chassi, rodízios centralizadores, meios de limpeza, e
comando a distância, além de meios capazes de alterar o seu
posicionamento angular sobre o \textit{riser}, em que:

- O chassi estrutural é preferencialmente conformado por uma
armação treliçada com formato de concepção livre, desde que a
configuração construtiva final seja capaz de envolver um segmento
do riser a ser limpo; serve de suporte para afixar e abarcar em seu
interior os rodízios centralizadores, os meios de limpeza e
acessórios; 

- Os rodízios centralizadores são pelo menos na quantidade de um
par, e são afixados no chassi, contrapostos entre si em um mesmo
plano horizontal, de forma que após posicionar o chassi estrutural
em torno do \textit{riser}, os referidos rodízios contraponham-se à parede
do riser forçando o posicionamento do chassi para que o seu eixo
vertical fique concêntrico ao eixo vertical do segmento do \textit{riser} por
ele envolvido

- Meios de limpeza que compreende uma aparelhagem de
escovação, formada por dois conjuntos de pares de escovas
cilíndricas, contrapostas entre si, de modo a manter contato
permanente contra lados opostos da parede externa no \textit{riser}


\vspace{1,0cm}
 \begin{center}
  \textbf{US 6,174,381 B1} 
 
 \textbf{Método para limpeza, principalmente de resídos sólidos, na parte interna de tubos}
 \end{center}
 
 \textbf{Resumo:}
 
 Data da publicação:16/01/2001
 
 Empresa: N/A
 
 \vspace{1,0cm}

 A base da presente invenção se da pela problemática de identificar um método de limpeza efetiva de sedimentos sólidos de carbono, dentro de tubos, sem causar danos a superfície do mesmo, além de elaborar um dispositivo que seja capaz de limpar tubulações de pequenos diâmetros.
 
 A metodologia orienta alguns passos para um resultado apropriado.
 
 - Inserção do elemento de limpeza dentro dos tubos
 
 - Rotação do componente de limpeza, simultaneamente a inserção de agua na cavidade, movimentando este pelo curso do tubo para remoção de resíduos sólidos
 
 - Rotacionar o componente de limpeza ao mínimo de 1000RPM
 
 - Rotações em movimento horário e anti-horário
 
 Em relação ao dispositivo que executará a atividade de limpeza, é orientado que:
 
 - O dispositivo compreenda um alojamento "\textit{Housing}", com um dispositivo de limpeza acomplado a esta.
 
 - O dispositivo é confeccionado em um formato roliçõ, onde haja movimentação rotacional pelo seu próprio eixo.
 
  " 
O método de limpeza é executado por um dispositivo especialmente
 feito para sua realização. Compreende pelo menos dois rolos
 com ranhuras anulares fixadas na superfície da extremidade da
\textit{housing} com a capacidade de girar em torno de seu eixo"


\vspace{1,0cm}
 \begin{center}
  \textbf{US 2008/0135232 A1} 
 
 \textbf{Ferramente de limpeza de conduíte submerso}
 \end{center}
 
 \textbf{Resumo:}
 
 Data da publicação:12/06/2008
 
 Empresa: OCEANEERING INT INC
 
 \vspace{1,0cm}
 
 Esta invenção é direcionada para limpeza de um conduíte submerso. O conduíte submerso pode admitir várias formas, incluindo mas nao se limitando à, \textit{risers}.
 
 A ferramente em questão é composta por:
 
 - Um colar de fixação dimensionado para prender em um conduíte
 
 - Uma escova acoplada ao primeiro eixo
 
 - Um suporte de montagem compreendendo a extremidade da primeira região, uma segunda região oposta a primeira, e dois suportes de montagem, o primeiro anexado a superficie da placa inferior do colar. E  compreendendo ,ainda, um braço de montagem que se estende de cada uma de suas extremidades, cada um dos referidos braços compreende um canal de eixo.
 
 - Um primeiro recebedor de fluido hidráulico acoplado a primeira extremidade do primeiro eixo, este primeiro, composto por 2 aberturas de fluido.



\vspace{1,0cm}
 \begin{center}
  \textbf{WO 2015/171874 Al} 
 
 \textbf{Aparelho e métedo de limpeza de estruturas submersas}
 \end{center}
 
 \textbf{Resumo:}
 
 Data da publicação:12/11/2015
 
 Empresa: OCEANEERING INT INC
 
 \vspace{1,0cm} 
 
 A patente acima lista as possibilidades e metodos de operacionalização de inspeção, limpeza, com (ROV), que neste caso é configurado e implantado em uma embarcação de apoio.
 
 As estruturas básicas do ROV são uma "\textit{housing}" com perfil hidrodinâmico; sistema de propulsão configurados com 1 ou mais motores elétricos; Bomba d'água; conformado para ter uma flutuabilidade ligeriramente negativa; configurado para ter autonomia em  longas distâncias do barco de apoio;; sistema de fluatação de emergência; ter a capacidade a acomplar em sua estrutura um manipulador;  
 

\vspace{1,0cm}
\hspace{-0,75cm}
\textbf{A importância dos reparos}
\vspace{1,0cm}
 
Como ja citado, as estruturas submersas das operações de O\&G (óleo e Gás) estão sujeitas a diversas intempéries.
 
\end{document}
